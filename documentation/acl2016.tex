%
% File acl2016.tex
%
%% Based on the style files for ACL-2015, with some improvements
%%  taken from the NAACL-2016 style
%% Based on the style files for ACL-2014, which were, in turn,
%% Based on the style files for ACL-2013, which were, in turn,
%% Based on the style files for ACL-2012, which were, in turn,
%% based on the style files for ACL-2011, which were, in turn, 
%% based on the style files for ACL-2010, which were, in turn, 
%% based on the style files for ACL-IJCNLP-2009, which were, in turn,
%% based on the style files for EACL-2009 and IJCNLP-2008...

%% Based on the style files for EACL 2006 by 
%%e.agirre@ehu.es or Sergi.Balari@uab.es
%% and that of ACL 08 by Joakim Nivre and Noah Smith

\documentclass[11pt]{article}
\usepackage{acl2016}
\usepackage{times}
\usepackage{url}
\usepackage{latexsym}
\usepackage[hidelinks]{hyperref}
%\aclfinalcopy % Uncomment this line for the final submission
%\def\aclpaperid{***} %  Enter the acl Paper ID here

%\setlength\titlebox{5cm}
% You can expand the titlebox if you need extra space
% to show all the authors. Please do not make the titlebox
% smaller than 5cm (the original size); we will check this
% in the camera-ready version and ask you to change it back.

\newcommand\BibTeX{B{\sc ib}\TeX}

\title{Building predictors for the game mia}
%nur sichtbar in finaler submission.
\author{First Author \\
  Affiliation / Address line 1 \\
  Affiliation / Address line 2 \\
  Affiliation / Address line 3 \\
  {\tt email@domain} \\\And
  Alexander Diegel \\
  Affiliation / Address line 1 \\
  Affiliation / Address line 2 \\
  Affiliation / Address line 3 \\
  {\tt email@domain} \\}

\date{}

\begin{document}
\maketitle
\begin{abstract}
  %TODO
\end{abstract}

\section{Credits}

This document has been adapted from the instructions for earlier ACL
and NAACL proceedings, including those for NAACL-2016 by Margaret
Mitchell, ACL-2012 by Maggie Li and Michael
White, those from ACL-2010 by Jing-Shing Chang and Philipp Koehn,
those for ACL-2008 by Johanna D. Moore, Simone Teufel, James Allan,
and Sadaoki Furui, those for ACL-2005 by Hwee Tou Ng and Kemal
Oflazer, those for ACL-2002 by Eugene Charniak and Dekang Lin, and
earlier ACL and EACL formats. Those versions were written by several
people, including John Chen, Henry S. Thompson and Donald
Walker. Additional elements were taken from the formatting
instructions of the {\em International Joint Conference on Artificial
  Intelligence} and the \emph{Conference on Computer Vision and
Pattern Recognition}.

\section{Introduction}

%TODO



\section{The Game}
Mia is a simple dice game that is played with two dices and a flat bottomed container (or a dice cup). At the beginning each player has a certain amount of lives (e.g. five).
The first player rolls the dices but keeps their values hidden from the other players. He then can decide if he wants to tell the truth to the next player and announce a value that was was actually rolled. Alternatively he can lie and announce a greater or lesser value than the rolled one.
But each player has to announce a greater value than the previous player.

The next player (who still has not seen the actual values) can now believe the passer, call the passer a liar and look on the dice or pass the dice to the next player (still without looking) announcing a higher value. 

A player looses a life if he called the previous one a liar and looked on the values to find out that they are what the previous player has announced or even higher. Otherwise the previous player looses a life. 

The higher value of the roll is multiplied by then and then added to the other die (a 4 and a 2 is 42). 
The \textbf{scoring} is from highest to lowest:  21 (Mia), 11, 22, 33, 44, 55, 66, 65, 64, 63, 62, 61, 54, 53, 52, 51, 43, 42, 41, 32, 31.

If a player announces mia the next player either believes him, give up (without looking at the dices) and looses one life. Or he may look at the dice. If it was actually mia then he looses two lifes if it was not, the previous player looses a life. (For further information see \cite{mia:2016}.)
\section{The Strategies}

\subsection{Statistic Approach}
\subsection{Approach with certain degree of randomness}
\subsection{A SVM learning approach}

\section{The predictor}


%\begin{table}
%\centering
%\small
%\begin{tabular}{cc}
%\begin{tabular}{|l|l|}
%\hline
%{\bf Command} & {\bf Output}\\\hline
%\verb|{\"a}| & {\"a} \\
%\verb|{\^e}| & {\^e} \\
%\verb|{\`i}| & {\`i} \\ 
%\verb|{\.I}| & {\.I} \\ 
%\verb|{\o}| & {\o} \\
%\verb|{\'u}| & {\'u}  \\ 
%\verb|{\aa}| & {\aa}  \\\hline
%\end{tabular} & 
%\begin{tabular}{|l|l|}
%\hline
%{\bf Command} & {\bf  Output}\\\hline
%\verb|{\c c}| & {\c c} \\ 
%\verb|{\u g}| & {\u g} \\ 
%\verb|{\l}| & {\l} \\ 
%\verb|{\~n}| & {\~n} \\ 
%\verb|{\H o}| & {\H o} \\ 
%\verb|{\v r}| & {\v r} \\ 
%\verb|{\ss}| & {\ss} \\\hline
%\end{tabular}
%\end{tabular}
%\caption{Example commands for accented characters, to be used in, e.g., \BibTeX\ names.}\label{tab:accents}
%\end{table}



%\penalty -5000

%We suggest that instead of
%\begin{quote}
%  ``\cite{Gusfield:97} showed that ...''
%\end{quote}
%you use
%\begin{quote}
%``Gusfield \shortcite{Gusfield:97}   showed that ...''
%\end{quote}




\section*{Acknowledgments}

The acknowledgments should go immediately before the references.  Do
not number the acknowledgments section. Do not include this section
when submitting your paper for review.

% include your own bib file like this:
%\bibliographystyle{acl}
%\bibliography{acl2016}
\bibliography{acl2016}
\bibliographystyle{acl2016}
\appendix

\section{Supplemental Material}
\label{sec:supplemental}
%ACL 2016 also encourages the submission of supplementary material
%to report preprocessing decisions, model parameters, and other details
%necessary for the replication of the experiments reported in the 
%paper. Seemingly small preprocessing decisions can sometimes make
%a large difference in performance, so it is crucial to record such
%decisions to precisely characterize state-of-the-art methods.
%
%Nonetheless, supplementary material should be supplementary (rather
%than central) to the paper. It may include explanations or details
%of proofs or deriations that do not fit into the paper, lists of
%features or feature tempates, sample inputs and outputs for a system,
%pseudo-code or source code, and data. (Source code and data should
%be separate uploads, rather than part of the paper).
%
%The paper should not rely on the supplementary material: while the paper
%may refer to and cite the supplementary material will be available to the
%reviewers, they will not be asked to review the
%supplementary material.
%
%Appendices (i.e. supplementary material in the form of proofs, tables,
%or pseudo-code) should come after the references, as shown here. Use
%\verb|\appendix| before any appendix section to switch the section
%numbering over to letters.


\end{document}
